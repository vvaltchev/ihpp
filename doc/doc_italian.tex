
\documentclass[a4paper,11pt]{report}

\author{Vladislav K. Valtchev} 
\title{k-Calling Context Profiling Pintool}

\usepackage[utf8]{inputenc}
\usepackage[T1]{fontenc}

\usepackage[italian]{babel}

\pagestyle{headings}

\begin{document}

\maketitle

\chapter{Introduzione}
\section{Le prestazioni di un software}

Ci sono diversi modi per analizzare le prestazioni di un programma software perché naturalmente
esistono diverse angolazioni dalle quali è possibile osservare il problema stesso: 
c'è la quantità di memoria utilizzata (picco, massima, minima, media), la quantità 
di scritture/letture su disco, il numero di operazioni di rete, ma sicuramente
al primo posto c'è la misurazione del \textbf{tempo di esecuzione}. 
Esso in qualche modo riassume tutti gli indici prestazionali di un software: 
se un programma svolge un lavoro in un tempo che, relativamente alla nostra percezione umana,
sembra buono, spesso ci sentiamo abbastanza soddisfatti e proseguiamo lavorando su altro.
Questo modo di procedere sempre più spesso porta alla produzione di cattivo codice: l'aumento esponenziale della potenza delle CPU permette sempre più al "cattivo codice" di passare 
inosservato. Ad esempio, una operazione della durata di 0,1s (tempo CPU) che a noi sembra "veloce" 
in realtà comporta l'esecuzione di centinaia di milioni di istruzioni; ma il programma aveva 
veramente il bisogno di eseguirle tutte quante quelle istruzioni oppure poteva essere \emph{ottimizzato}?

\end{document}