
\documentclass[a4paper,11pt]{report}

\author{Vladislav K. Valtchev} 
\title{IHPP: An Intraprocedural Hot Path Profiler}


\usepackage[utf8]{inputenc}
\usepackage[english]{babel}

\usepackage{hyperref}
\usepackage{xcolor,graphicx}
\usepackage{mdwlist}
\usepackage{fix-cm}

\hypersetup{
    colorlinks,
    citecolor=black,
    filecolor=black,
    linkcolor=black,
    urlcolor=black
}

%in order to make the analytical index
%\usepackage{makeidx}
%\makeindex

\pagestyle{headings}

\begin{document}

\thispagestyle{empty}

\begin{figure}
\centering
\includegraphics[scale=0.6]{logo}
\end{figure}


\begin{center}


{\Large\textsc{Universit\`a degli studi di Roma}\\} 
{\huge\textsc{La Sapienza}\\[10pt]}
{\huge\textsc{Facolt\`a di Ingegneria}\\[40pt]} 

{\large Tesi di laurea in: \\}
{\LARGE\textsc{Ingegneria Informatica}\\[50pt]}

{\large Docente relatore: \\}
{\large Prof. Camil Demetrescu\\[20pt]}

{\large Candidato: \\}
{\large \textbf{Vladislav K. Valtchev}\\[40pt]}

{\large Anno accademico: 2011/2012\\}


\end{center}

%in order to print the analytical index
%\printindex

\pagebreak

\thispagestyle{empty}

\begin{center}

\vspace*{4.5cm}

\fontsize{70}{90}\selectfont \textbf{IHPP}\\
\fontsize{20}{35}\selectfont
\textit{An Intraprocedural Hot Path Profiler}\\

\vspace{11cm}

\fontsize{14}{20}\selectfont
\textbf{Vladislav K. Valtchev}

\end{center}
\pagebreak

\begin{abstract}
dwehkhjrkweh rlkewrhwe lkrhwek rhwel krhjweklrh welkrh wekrhwe krhwr
werwerwekr h jwekj rhwk jrhwk rhwl kjrh k kqhe qe wrkjehr r hk rhj
wlhrwelkjrfh sd hf kjh sdfkh jq ql lkqjhekqj q eqhkjeqkj kjahq 
erhjwe skafhd lkaerh welrk liwer  iwerh wit wielr qierhu lqr 
dwehkhjrkweh rlkewrhwe lkrhwek rhwel krhjweklrh welkrh wekrhwe krhwr
werwerwekr h jwekj rhwk jrhwk rhwl kjrh k kqhe qe wrkjehr r hk rhj
wlhrwelkjrfh sd hf kjh sdfkh jq ql lkqjhekqj q eqhkjeqkj kjahq 
erhjwe skafhd lkaerh welrk liwer  iwerh wit wielr qierhu lqr 
dwehkhjrkweh rlkewrhwe lkrhwek rhwel krhjweklrh welkrh wekrhwe krhwr
werwerwekr h jwekj rhwk jrhwk rhwl kjrh k kqhe qe wrkjehr r hk rhj
wlhrwelkjrfh sd hf kjh sdfkh jq ql lkqjhekqj q eqhkjeqkj kjahq 
erhjwe skafhd lkaerh welrk liwer  iwerh wit wielr qierhu lqr 
dwehkhjrkweh rlkewrhwe lkrhwek rhwel krhjweklrh welkrh wekrhwe krhwr
werwerwekr h jwekj rhwk jrhwk rhwl kjrh k kqhe qe wrkjehr r hk rhj

\end{abstract}


\tableofcontents

\chapter{Introduction}


\section{The context (state of art)}
\section{Motivations}
\section{Contributions}
\section{Thesis structure}

\chapter{Program analysis}

Program analysis is the process of analyzing the behavior of computer programs\footnote{Program analysis: \url{http://en.wikipedia.org/wiki/Program_analysis}}.
There are two main approaches in program analysis: static and dynamic analysis.
The main difference between them is that in \emph{static} analysis nothing is executed: the analysis
is made only by observing the program source code or the compiled program instructions. Instead, the \emph{dynamic} program analysis is based on executing the program and ``watching'' what exactly is it doing, even in real time if possible.

\section{Static analysis}

Static analysis can be made either by hand or by using another program.


\section{Dynamic analysis}

\subsection{The profiling}



\chapter{The Approach: An Intraprocedural Hot Path Profiler}


\section{Algorithms and data structures used}

\chapter{The implementation}


\chapter{Evalutation: Case studies}


\chapter{Conclusions}

\end{document}